\documentclass[twocolumn,11pt]{article} 
%\usepackage[spanish]{babel} 
%\selectlanguage{english}  
\usepackage[utf8]{inputenc} 
\usepackage[T1]{fontenc} 
\usepackage{array}
\usepackage{geometry}
 \geometry{
 a4paper,
 total={170mm,257mm},
 left=20mm,
 top=20mm,
 }
\usepackage{amsmath, amsthm, amsfonts} 
\usepackage{graphicx} 
\usepackage[colorinlistoftodos]{todonotes} 
\usepackage[colorlinks=true, allcolors=blue]{hyperref} 
\usepackage[square,numbers]{natbib}
\usepackage{upgreek}
\usepackage{lipsum}  

\title{Segundo Laboratorio: \\ Magnetismo} 
\author{Benjam\'in Ayala Baeza  \\ \href{mailto:beayala2021@udec.cl}{beayala2021@udec.cl} 
        \and Mar\'ia Ignacia Espinoza Inzunza \\ \href{mailto:marespinoza2021@udec.cl}{marespinoza2021@udec.cl} \\
\date{Universidad de Concepci\'on \\ \today}
}
\setlength{\marginparwidth}{2cm}
\begin{document}
\twocolumn[
  \begin{@twocolumnfalse}
    \maketitle
    \begin{abstract}
     En este documento veremos como se ve y hace uso de leyes del electromagnetismo en una serie de experimentos que pueden ser realizados con materiales que podemos encontrar en nuestro hogar. A partir de estos experimentos expondremos datos y conclusiones que mostrar\'an cierto uso de recursos f\'isicos en la vida y pruebas simples de una demostraci\'on de estas leyes f\'isicas.
    \end{abstract}
  \end{@twocolumnfalse}
]

\section{Introducci\'on} \label{intro}
        \lipsum[1-2]



\section{Marco Te\'orico}
        \lipsum[1-2]


\section{Experimentaci\'on}
        \lipsum[1-2]


\section{An\'alisis y Resultados}
        \lipsum[1-2]


\section{Conclusi\'on}
        \lipsum[1]


\bibliographystyle{apalike}
\bibliography{weas.bib}
\end{document}